\capitulo{4}{Técnicas y herramientas}

\section{Metodologías}\label{metodologias}

\subsection{Scrum}\label{scrum}
Para gestionar el proyecto, se utilizó Scrum, que es un enfoque ágil que permite trabajar por etapas cortas (llamadas \emph{sprints}). Así, se avanza paso a paso, revisando y mejorando el producto en cada fase. Esto ayuda a adaptarse rápidamente a los cambios y a mantener el trabajo bien organizado.

\section{Gestión del proyecto}\label{gestion-del-proyecto}

\subsection{Control de versiones}\label{control-de-versiones}

\begin{itemize}
\tightlist
\item
  Herramientas consideradas: \href{https://git-scm.com/}{Git} y
  \href{https://subversion.apache.org/}{Subversion}.
\item
  Opción elegida: \href{https://git-scm.com/}{Git}.
\end{itemize}

En este proyecto, se decidió utilizar Git porque es el sistema de control de versiones más utilizado y, por experiencia, el que mejor se adapta a cualquier tipo de desarrollo. Lo que ha resultado más útil es que permite crear ramas para probar nuevas funcionalidades sin tocar el código principal; es decir, si algo sale mal, no afecta a lo que ya funciona. Otra ventaja es que, al ser distribuido, en un equipo con varios integrantes, cada uno tiene una copia completa del proyecto en su máquina, permitiendo seguir trabajando incluso cuando haya problemas con la conexión a Internet. Además, como cliente de Git, se ha utilizado TortoiseGit~\cite{tortoisegit} para todo lo referente a la documentación del proyecto.

\subsection{Repositorio}\label{repositorio}

\begin{itemize}
\tightlist
\item
  Herramientas consideradas: \href{https://github.com/}{GitHub},
  \href{https://bitbucket.org/}{Bitbucket} y
  \href{https://gitlab.com/}{GitLab}.
\item
  Opción elegida: \href{https://github.com/}{GitHub}.
\end{itemize}

En este proyecto, se optó por usar GitHub porque, además de basarse en Git, ofrece herramientas extra que han facilitado mucho el trabajo. Por un lado, la integración con CI/CD~\cite{ci-cd} permitió automatizar pruebas y despliegues, algo que el tutor pidió demostrar como parte de las buenas prácticas. Por ejemplo, cada vez que se subía código a la rama principal, se ejecutaban automáticamente los \emph{tests}, lo que ayudó a detectar errores antes de la entrega. También ofrece el uso de los \emph{issues} de GitHub para organizar las tareas: se asignaban etiquetas según el tipo de trabajo (\emph{bug}, mejora, documentación, etc.). Así, de un simple vistazo siempre se puede saber en qué se estaba trabajando en cada momento. Otra cosa que resultó útil fue la opción de marcar versiones importantes con \emph{tags}, sobre todo cuando se terminaba una fase del proyecto y se quería guardar un ``punto de control'' antes de seguir avanzando.

\subsection{Integración continua y análisis de calidad}\label{integracion-continua-analisis-calidad}
GitHub Actions por su integración nativa con el repositorio permite automatizar el \emph{pipeline} de CI/CD directamente desde GitHub. Su facilidad de configuración mediante archivos YAML~\cite{yaml} permite habilitar la ejecución de diferentes \emph{workflows} para la compilación y empaquetado del proyecto, así como la comunicación con SonarCloud para el análisis estático del código para identificar bugs, vulnerabilidades y \emph{code smells} automáticamente. Su integración con GitHub Actions para análisis continuo en cada \emph{commit}, y el hecho de proporcionar métricas objetivas de calidad (cobertura de código, complejidad ciclomática, duplicación de código, etc...) enriquecen significativamente la documentación técnica del proyecto, ofreciendo un enfoque profesional de cara a la calidad del software.

\subsection{Documentación}\label{documentacion}
Para documentar el proyecto, se utilizó \LaTeX{} a través de la plataforma Overleaf~\cite{overleaf}. Esta combinación fue propuesta por los tutores, ya que permite generar documentos académicos con un formato profesional y una tipografía de alta calidad, aspectos esenciales para cumplir con los requisitos de presentación. Al estar basado en la nube, Overleaf facilita el trabajo colaborativo en tiempo real desde cualquier dispositivo, lo que simplificó las revisiones con los tutores. Además, la gestión automática de referencias bibliográficas y la numeración de secciones redujeron significativamente los errores manuales, asegurando que el documento final cumpliera con los estándares académicos establecidos.

\section{Patrones de diseño}\label{patron-de-diseño}

El sistema ha sido diseñado siguiendo principios de arquitectura en capas, separación de responsabilidades y buenas prácticas de desarrollo. A continuación, se describen algunos de los patrones utilizados.

\subsection{Patrón Modelo-Vista-Controlador (MVC)}\label{patron-modelo-vista-controlador}
Para el desarrollo de la aplicación web de gestión de reservas con Spring Boot, se siguió el patrón MVC por su separación clara de responsabilidades que facilita el mantenimiento y escalabilidad del código al dividir la lógica de negocio (Modelo), la presentación (Vista) y el control de flujo (Controlador). Su integración nativa con Spring Boot mediante las anotaciones que ofrece el \emph{framework} simplifica significativamente el desarrollo. Además, ofrece flexibilidad para cambios futuros como modificar la interfaz de usuario sin afectar la lógica de negocio, o modificar la lógica de negocio sin afectar a la capa visual o al modelo de datos.

\subsection{Arquitectura por Capas}\label{arquitectura-por-capas}
De alguna forma, o bien como consecuencia o bien como complementación al punto anterior, el patrón MVC favorece la construcción de los diferentes elementos separándose en diferentes capas, donde cada capa (presentación, lógica de negocio, acceso a datos) se alinea naturalmente con los componentes MVC, creando una estructura coherente y bien organizada. Esta estructura proporciona alta cohesión y bajo acoplamiento, permitiendo que los cambios en una capa (como modificar la base de datos) no afecten a las demás, mientras que MVC organiza la interacción usuario-sistema dentro de cada capa. Además, esta separación facilita la implementación de principios SOLID~\cite{solid} posibilitando el \emph{testing} independiente de cada capa, donde se puede probar la lógica de negocio sin depender de la interfaz de usuario o de la base de datos. Esta combinación entre arquitectura por capas y MVC es una buena práctica que conduce a un diseño maduro y profesional que cumple con los estándares para el desarrollo de aplicaciones.

\subsection{Patrón Repository}\label{patron-repository}
Al disponer Spring Boot de una integración nativa con Spring Data JPA, se permite la creación de repositorios automáticamente mediante interfaces, reduciendo significativamente el código. Este patrón abstrae la lógica de acceso a los datos de la capa de negocio, permitiendo cambiar la implementación de la capa de persistencia sin afectar el resto de la aplicación y facilitando el \emph{testing} mediante \emph{mocks}. Además, Spring Boot proporciona funcionalidades avanzadas como \emph{query methods} automáticos y paginación. El uso de este patrón garantiza el cumplimiento del principio de inversión de dependencias, creando una arquitectura limpia y mantenible.
\imagen{patron-repository2}{Patrón Repository.}{1}

\subsection{Patrón Service Layer}\label{patron-service-layer}
Gracias a la utilización del patrón de diseño MVC comentado anteriormente, la implementación conduce casi de forma intuitiva o directa al uso del patrón \emph{service layer} que actúa como intermediario entre los \emph{Controllers} y los \emph{Repositories}, encapsulando toda la lógica de negocio y las reglas de validación en una capa dedicada que mantiene los controladores enfocados únicamente al manejo de peticiones HTTP. Su integración con Spring Boot mediante \emph{@Service} facilita la inyección de dependencias y la gestión transaccional con \emph{@Transactional}, permitiendo la reutilización de la lógica de negocio desde diferentes controladores o incluso APIs REST y servicios web. Este patrón refuerza la separación de responsabilidades del MVC, donde el Controller delega las operaciones complejas en el \emph{Service}, que a su vez utiliza los \emph{Repositories} para la persistencia, creando una arquitectura limpia, testeable y que cumple con los principios de buenas prácticas del diseño de software.

\subsection{Patrón Data Transfer Object (DTO)}\label{patron-data-transfer-object}
Con el uso del patrón DTO se consigue transferir datos entre capas sin exponer la estructura interna de las entidades JPA, evitando problemas como la serialización de relaciones \emph{lazy} y mejorando la seguridad al controlar qué información se envía al cliente. Los DTOs proporcionan flexibilidad en la representación de datos, permitiendo combinar información de múltiples entidades o mostrar vistas específicas según el contexto. Por ello, se facilita la evolución independiente de la API y el modelo de datos, donde se pueden modificar las entidades de la base de datos sin afectar los contratos de la API, y mejorando el rendimiento al reducir la cantidad de datos transferidos.

\section{Librerías}\label{librerias}

\subsection{JUnit}\label{junit}
JUnit es considerado el \emph{framework} de \emph{testing} por excelencia, situándose como un estándar de facto para los \emph{tests} en Java. Viene integrado nativamente con Spring Boot Test, facilitando la escritura de tests unitarios y de integración sin tener que realizar una configuración adicional. Su sintaxis intuitiva con anotaciones permite crear tests legibles y mantenibles, posibilitando el testing de controladores y servicios de forma aislada. Además, JUnit proporciona reporting detallado de resultados y se integra perfectamente con herramientas de CI/CD como GitHub Actions y análisis de cobertura con SonarCloud, permitiendo obtener métricas de cobertura de la calidad del software.

\subsection{Mockito}\label{mockito}
Para complementar los \emph{tests} unitarios implementados con JUnit, se ha utilizado la librería Mockito por su capacidad de crear \emph{mocks} y \emph{stubs} de dependencias que permiten aislar completamente los elementos que se están \emph{testeando}, resultando especialmente útil para testear servicios sin depender de \emph{repositories} reales o bases de datos. Su integración con JUnit mediante anotaciones como \emph{@Mock} y \emph{@InjectMocks} simplifica la configuración de tests, mientras que su sintaxis intuitiva con métodos como \emph{when().thenReturn()} hace que los \emph{tests} sean legibles y entendibles. Con el uso de Mockito se ha posibilitado el testing de las diferentes capas que componen la arquitectura, permitiendo verificar interacciones entre componentes (\emph{verify()}) y simular diferentes escenarios de error o éxito sin la complejidad de configurar un entorno completo, lo que resulta práctico para demostrar buenas prácticas de testing en un proyecto.

\newpage

\section{Desarrollo Web}\label{desarrollo-web}

\subsection{Thymeleaf}\label{thymeleaf}
Por su integración nativa y oficial con Spring Boot, se ha utilizado Thymeleaf ya que elimina configuraciones adicionales y proporciona soporte completo para el patrón MVC mediante resolución automática de plantillas y \emph{binding} de modelos. Su sintaxis natural en HTML permite que las plantillas sean visualizables en navegadores sin procesamiento, facilitando el desarrollo colaborativo con diseñadores, mientras que sus expresiones Spring EL se integran a la perfección con los objetos del modelo pasados desde los controladores. Además, Thymeleaf ofrece características específicas para aplicaciones web como validación de formularios integrada con Spring Validation, internacionalización automática, y fragmentos reutilizables, creando una solución completa que demuestra un uso adecuado del ecosistema Spring para el desarrollo de aplicaciones web.

\subsection{Bootstrap}\label{bootstrap}
Como complementación de Thymeleaf, se ha utilizado Bootstrap por su integración sencilla con sus plantillas o los recursos estáticos de Spring Boot, permitiendo aplicar clases CSS directamente en los elementos HTML, así como crear interfaces responsivas sin una configuración compleja. Su sistema de componentes predefinidos (formularios, tablas, modales) se adapta a la perfección a las funcionalidades utilizadas en la aplicación, mientras que la gestión de recursos estáticos de Spring Boot permite servir Bootstrap de forma optimizada y cacheable. Además, Bootstrap proporciona consistencia visual y experiencia de usuario sin requerir conocimientos avanzados de CSS, permitiendo que el proyecto se enfoque en la lógica de negocio y la arquitectura del \emph{backend} manteniendo una interfaz moderna y funcional.

\subsection{JavaScript}\label{javascript}
Se ha utilizado JavaScript por su capacidad para añadir interactividad dinámica en aquellos cometidos en los que Bootstrap no puede proporcionar por sí solo, como validación de formularios en tiempo real, calendarios interactivos para selección de fechas de reserva, y componentes dinámicos que mejoren la experiencia de usuario sin recargar la página. Su integración natural con Thymeleaf permite generar código JavaScript dinámico con datos del servidor, mientras que se complementa perfectamente con los componentes de Bootstrap añadiendo funcionalidad a modales, \emph{dropdowns}, y formularios mediante \emph{event listeners} y manipulación del DOM\footnote{DOM es la sigla de document object model}. Además, JavaScript permite comunicación asíncrona con el \emph{backend} mediante AJAX~\cite{ajax} para funcionalidades como verificación de disponibilidad de reservas en tiempo real, creando una aplicación web moderna e interactiva.

\subsection{CSS}\label{css}
Se ha utilizado un CSS personalizado para permitir su adaptación y extensión de los estilos de Bootstrap sin modificar el \emph{framework} base, consiguiendo crear una identidad visual propia para la aplicación mediante variables CSS personalizadas y una sobrescritura selectiva de clases Bootstrap. Su integración con Thymeleaf permite aplicar estilos condicionales y crear hojas de estilo específicas para diferentes vistas de la aplicación de reservas. Además, CSS permite optimizar la experiencia de usuario con animaciones o transiciones que complementen la \emph{responsividad} de Bootstrap, a la vez que facilita la personalización específica de elementos visuales que Bootstrap no cubre.

\section{Entorno de desarrollo integrado (IDE)}\label{entorno-de-desarrollo-integrado}

\begin{itemize}
\tightlist
\item
  Herramientas consideradas: 
  \href{https://code.visualstudio.com/}{Visual Studio code}, 
  \href{https://eclipse.org/}{Eclipse} y
  \href{https://www.jetbrains.com/idea/}{IntelliJ IDEA}.
\item
  Opción elegida: \href{https://eclipse.org/}{Eclipse}.
\end{itemize}

Para la codificación de la aplicación, se ha utilizado Eclipse como IDE de desarrollo por su integración nativa con el ecosistema Java empresarial y su excelente soporte para proyectos Maven/Gradle, \emph{debugging} avanzado, y herramientas específicas de Spring como Spring Tool Suite (STS) que facilitan el desarrollo con autocompletado inteligente y configuración automática. Su gestión robusta de proyectos grandes con refactoring automático, navegación de código eficiente, e integración con sistemas de control de versiones como Git lo hacen adecuado para la organización y mantenibilidad del código a largo plazo. Además, Eclipse ofrece plugins especializados para tecnologías del proyecto como Thymeleaf, JUnit, asistentes a la codificación como GitHub Copilot~\cite{github-copilot} y herramientas de análisis de código, mientras que por su naturaleza gratuita y \emph{open-source} lo convierten en una opción accesible para cualquier desarrollador para su uso como herramienta profesional estándar de la industria Java sin costes adicionales.
