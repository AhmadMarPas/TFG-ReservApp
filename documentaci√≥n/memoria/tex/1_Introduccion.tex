\capitulo{1}{Introducción}

En el entorno empresarial actual, la gestión eficiente de los espacios de trabajo es necesaria para garantizar una organización adecuada y un uso eficiente de los recursos disponibles. En muchas empresas, las salas de reuniones son espacios que se utilizan para la coordinación de equipos, la realización de presentaciones y la toma de decisiones estratégicas. Sin embargo, la falta de un sistema apropiado para la reserva de estas salas puede generar problemas como la ocupación simultánea, la falta de visibilidad sobre su disponibilidad y la dificultad para gestionar cambios o cancelaciones.

En mi lugar de trabajo, este problema es recurrente, ya que actualmente la asignación de salas de reuniones se realiza de manera informal, lo que provoca confusión entre los empleados, pérdida de tiempo y conflictos en la planificación de reuniones. Para abordar esta problemática, se propone el desarrollo de una aplicación web que facilite la gestión de reservas de salas de reuniones de manera eficiente, organizada y accesible para todos los empleados.

La implementación de soluciones software para la gestión de espacios se alinea con los principios de la gestión sostenible de instalaciones. Como señalan [Autor et al., 2023]~\cite{su15043174} en su revisión sobre gestión sostenible de instalaciones, las instalaciones adecuadamente gestionadas no solo contribuyen al ahorro de costos, sino que también mejoran la moral y eficiencia de los empleados. En este contexto, un sistema de reservas de salas de reuniones no solo optimiza el uso de los espacios físicos disponibles, sino que también reduce el desperdicio de recursos energéticos asociados a salas reservadas pero no utilizadas, mientras mejora la experiencia laboral de los empleados al eliminar incertidumbres y conflictos en la planificación de reuniones.

Esta aplicación proporcionará una plataforma centralizada en la que los usuarios podrán consultar la disponibilidad de las salas en tiempo real, realizar reservas de manera sencilla y gestionar modificaciones o cancelaciones sin generar conflictos. Además, se implementarán restricciones que evitarán reservas superpuestas o solapadas, garantizando un uso óptimo de los espacios.

El objetivo principal de este Trabajo de Fin de Grado es diseñar e implementar una solución tecnológica que optimice la gestión de las salas de reuniones, mejorando la eficiencia en la organización interna de la empresa. A lo largo de este documento, se detallará el proceso de desarrollo de la aplicación, incluyendo el análisis del problema, el diseño de la solución, la implementación y la evaluación del sistema propuesto.

\section{Estructura de la memoria}\label{estructura-de-la-memoria}

La memoria sigue la siguiente estructura:

\begin{itemize}
\tightlist
\item
  \textbf{Introducción:} breve descripción del proyecto, estructura de la memoria y listado de materiales adjuntos.
\item
  \textbf{Objetivos del proyecto:} descripción de los objetivos que persigue el proyecto.
\item
  \textbf{Conceptos teóricos:} breve explicación de los conceptos necesarios para la realización del proyecto.
\item
  \textbf{Técnicas y herramientas:} descripción de las metodologías y herramientas que han sido utilizadas para llevar a cabo el proyecto.
\item
  \textbf{Aspectos relevantes del desarrollo:} aspectos a destacar a lo largo de la realización del proyecto.
\item
  \textbf{Trabajos relacionados:} resumen de trabajos y proyectos ya realizados en el campo del proyecto en curso.
\item
  \textbf{Conclusiones y líneas de trabajo futuras:} conclusiones obtenidas al finalizar el proyecto y posibles ideas de continuidad.
\end{itemize}

Acompañando a esta memoria, se adjuntan los anexos que se detallan a continuación:

\begin{itemize}
\tightlist
\item
  \textbf{Plan del proyecto software:} presenta la planificación temporal y el estudio de viabilidad asociados al proyecto.
\item
  \textbf{Especificación de requisitos del software:} se documenta la fase de análisis, que abarca los objetivos generales, el catálogo de requisitos del sistema, y la definición de requisitos funcionales y no funcionales.
\item
  \textbf{Especificación de diseño:} se centra en la fase de diseño, cubriendo el alcance del software, el diseño de la base de datos, el diseño procedimental y la arquitectura general.
\item
  \textbf{Manual del programador:} recurso en el que se resumen los aspectos técnicos más relevantes del código fuente, como su organización, procesos de compilación, instalación, ejecución y procedimientos de prueba.
\item
  \textbf{Manual de usuario:} guía completa diseñada para facilitar el uso correcto de la aplicación por parte del usuario final.
\end{itemize}

\section{Recursos disponibles}\label{materiales-adjuntos}

Los siguientes recursos están accesibles a través de Internet: 

\begin{itemize}
\tightlist
\item
	Dirección del repositorio del proyecto~\cite{reservapp:repo}.
\item
	Dirección de la documentación del proyecto~\cite{reservapp:documentacion}.
\item	
	Dirección de las métricas del proyecto~\cite{reservapp:sonarcloud}.
\item	
    Dirección de la web de Reservas usada en el desarrollo~\cite{reservapp:app}.
\item
	Dirección del video con descripción del proyecto~\cite{reservapp:descripcion}.
\item	
	Dirección del video con demostración funcional de la aplicación~\cite{reservapp:demostracion}.
\end{itemize}
