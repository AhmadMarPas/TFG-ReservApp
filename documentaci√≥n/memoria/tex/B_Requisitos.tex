\apendice{Especificación de Requisitos}

\section{Introducción}
Este documento presenta la especificación completa de requisitos para ReservApp, un sistema web de gestión de reservas. El objetivo es definir de manera precisa y detallada las funcionalidades, características técnicas y restricciones del sistema, proporcionando una base sólida para el desarrollo, testing y mantenimiento de la aplicación.

ReservApp es una aplicación web diseñada para facilitar la gestión integral de reservas en establecimientos diversos. El sistema permite a los usuarios realizar reservas en diferentes establecimientos, gestionar convocatorias con otros usuarios, y a los administradores supervisar todo el proceso de reservas y gestión de usuarios.

\section{Objetivos generales}

El objetivo principal del actual TFG es el desarrollo de un sistema web de gestión de reservas que permita a los usuarios realizar reservas de manera eficiente en diferentes establecimientos, con capacidades avanzadas de convocatorias y administración centralizada. Y como consecuencia de ello, se deben contemplar los siguientes objetivos específicos:

\begin{itemize}
\tightlist
\item
Gestión de Establecimientos.
    \begin{itemize}
    \tightlist
    \item
    Permitir la configuración flexible de establecimientos con horarios específicos.
    \item
    Implementar sistema de aforo para controlar la capacidad de reservas.
    \item
    Proporcionar asignación de usuarios a establecimientos específicos.
    \item
    Facilitar la gestión administrativa de establecimientos.
    \end{itemize}

\item
Gestión de Usuarios.
    \begin{itemize}
    \tightlist
    \item
    Proporcionar un sistema de registro y autenticación seguro.
    \item
    Implementar control de acceso basado en roles (Usuario/Administrador).
    \item
    Permitir la gestión de perfiles de usuario personalizables.
    \end{itemize}

\item
Sistema de Reservas.
    \begin{itemize}
    \tightlist
    \item
    Ofrecer múltiples modalidades de reserva (slots predefinidos y horarios libres).
    \item
    Implementar validación automática de disponibilidad y conflictos.
    \item
    Proporcionar calendario interactivo para visualización de reservas.
    \item
    Permitir modificación y anulación de reservas con notificaciones
    \end{itemize}

\item
Sistema de Convocatorias
    \begin{itemize}
    \tightlist
    \item
    Facilitar la invitación de otros usuarios a reservas.
    \item
    Proporcionar gestión de enlaces de reunión y observaciones.
    \item
    Implementar búsqueda avanzada de usuarios para convocatorias.
    \item
    Automatizar notificaciones por correo electrónico.
    \end{itemize}

\item
Administración del Sistema
    \begin{itemize}
    \tightlist
    \item
    Proporcionar un panel de administración completo.
    \item
    Implementar gestión avanzada de usuarios (bloqueo, desbloqueo, eliminación).
    \item
    Ofrecer visualización de reservas por establecimiento y fecha.
    \end{itemize}

\item
Experiencia de Usuario.
    \begin{itemize}
    \tightlist
    \item
    Desarrollar interfaz intuitiva y responsiva.
    \item
    Implementar navegación eficiente entre funcionalidades.
    \item
    Optimizar rendimiento para carga rápida de páginas.
    \item
    Proporcionar feedback visual inmediato de acciones
    \end{itemize}

\end{itemize}




Una muestra de cómo podría ser una tabla de casos de uso:

% Caso de Uso 1 -> Consultar Experimentos.
\begin{table}[p]
	\centering
	\begin{tabularx}{\linewidth}{ p{0.21\columnwidth} p{0.71\columnwidth} }
		\toprule
		\textbf{CU-1}    & \textbf{Ejemplo de caso de uso}\\
		\toprule
		\textbf{Versión}              & 1.0    \\
		\textbf{Autor}                & Alumno \\
		\textbf{Requisitos asociados} & RF-xx, RF-xx \\
		\textbf{Descripción}          & La descripción del CU \\
		\textbf{Precondición}         & Precondiciones (podría haber más de una) \\
		\textbf{Acciones}             &
		\begin{enumerate}
			\def\labelenumi{\arabic{enumi}.}
			\tightlist
			\item Pasos del CU
			\item Pasos del CU (añadir tantos como sean necesarios)
		\end{enumerate}\\
		\textbf{Postcondición}        & Postcondiciones (podría haber más de una) \\
		\textbf{Excepciones}          & Excepciones \\
		\textbf{Importancia}          & Alta o Media o Baja... \\
		\bottomrule
	\end{tabularx}
	\caption{CU-1 Nombre del caso de uso.}
\end{table}

\section{Catálogo de requisitos}

\section{Especificación de requisitos}


