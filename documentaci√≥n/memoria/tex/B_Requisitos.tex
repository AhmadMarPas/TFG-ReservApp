\apendice{Especificación de Requisitos}

\section{Introducción}
Este documento presenta la especificación completa de requisitos para ReservApp, un sistema web de gestión de reservas. El objetivo es definir de manera precisa y detallada las funcionalidades, características técnicas y restricciones del sistema, proporcionando una base sólida para el desarrollo, testing y mantenimiento de la aplicación.

ReservApp es una aplicación web diseñada para facilitar la gestión integral de reservas en establecimientos diversos. El sistema permite a los usuarios realizar reservas en diferentes establecimientos, gestionar convocatorias con otros usuarios, y a los administradores supervisar todo el proceso de reservas y gestión de usuarios.



Una muestra de cómo podría ser una tabla de casos de uso:

% Caso de Uso 1 -> Consultar Experimentos.
\begin{table}[p]
	\centering
	\begin{tabularx}{\linewidth}{ p{0.21\columnwidth} p{0.71\columnwidth} }
		\toprule
		\textbf{CU-1}    & \textbf{Ejemplo de caso de uso}\\
		\toprule
		\textbf{Versión}              & 1.0    \\
		\textbf{Autor}                & Alumno \\
		\textbf{Requisitos asociados} & RF-xx, RF-xx \\
		\textbf{Descripción}          & La descripción del CU \\
		\textbf{Precondición}         & Precondiciones (podría haber más de una) \\
		\textbf{Acciones}             &
		\begin{enumerate}
			\def\labelenumi{\arabic{enumi}.}
			\tightlist
			\item Pasos del CU
			\item Pasos del CU (añadir tantos como sean necesarios)
		\end{enumerate}\\
		\textbf{Postcondición}        & Postcondiciones (podría haber más de una) \\
		\textbf{Excepciones}          & Excepciones \\
		\textbf{Importancia}          & Alta o Media o Baja... \\
		\bottomrule
	\end{tabularx}
	\caption{CU-1 Nombre del caso de uso.}
\end{table}

\section{Objetivos generales}

\section{Catálogo de requisitos}

\section{Especificación de requisitos}


