\capitulo{6}{Trabajos relacionados}

El desarrollo de sistemas de gestión de reservas ha sido un área de investigación y desarrollo activo durante las últimas décadas, especialmente con el auge de las tecnologías web y la digitalización de procesos empresariales. En este apartado se presenta un análisis de las soluciones más relevantes en el campo de la gestión de reservas de recursos, sistemas de planificación colaborativa y aplicaciones web empresariales que han servido como referencia y contexto para el desarrollo del presente proyecto.
La revisión de trabajos relacionados se ha estructurado en tres categorías principales: soluciones comerciales de gestión de reservas, proyectos académicos en el ámbito de la planificación de recursos, y \emph{frameworks} y metodologías de desarrollo de aplicaciones web empresariales. Esta categorización permite comprender tanto el estado actual del mercado como las contribuciones académicas que fundamentan las decisiones técnicas adoptadas en el proyecto.

\section{Soluciones comerciales de gestión de reservas}\label{soluciones-comerciales-gestion-reservas}
\subsection{Microsoft Bookings y Microsoft 365}\label{microsoft-booking-microsoft-365}
Microsoft Bookings~\cite{microsoft-booking} representa uno de los sistemas de reservas más extendidos en el ámbito empresarial, integrado nativamente en el ecosistema Microsoft 365~\cite{microsoft-365}. Este sistema destaca por su capacidad de sincronización automática con Outlook y su integración con Microsoft Teams para reuniones virtuales. Sin embargo, presenta limitaciones significativas en términos de personalización y flexibilidad, especialmente en organizaciones con necesidades específicas de gestión de espacios físicos.
La arquitectura de Microsoft Bookings sigue un modelo SaaS (Software as a Service)~\cite{saas} que, aunque reduce la complejidad de mantenimiento, limita las posibilidades de personalización y control sobre los datos. En contraste, el enfoque adoptado en el presente TFG permite una mayor flexibilidad arquitectónica y la posibilidad de adaptarse a requisitos específicos de la organización.

\subsection{Google Workspace y Calendar}\label{google-workspace-calendar}
El sistema de reservas de Google Workspace~\cite{google-workspace}, centrado en Google Calendar~\cite{google-calendar}, destaca la simplicidad de uso y la excelente integración con otros servicios de Google, pero presenta limitaciones en la gestión granular de recursos y la falta de funcionalidades específicas para la gestión de espacios físicos.
Una característica notable de Google Calendar es su API robusta que permite integraciones personalizadas, aspecto que ha influido en la decisión de diseñar la aplicación del presente TFG con una arquitectura modular que facilite futuras integraciones con sistemas externos.

\subsection{Robin y sistemas especializados}\label{robin-sistemas-especializados}
Entre las soluciones actualmente consideradas de nueva generación de sistemas especializados en la gestión de espacios de trabajo se encuentra Robin~\cite{robin}, especialmente populares tras la pandemia de COVID-19. Estos sistemas incorporan funcionalidades avanzadas como análisis de ocupación, integración con sistemas IoT y gestión de protocolos sanitarios.
Aunque Robin ofrece funcionalidades avanzadas, su modelo de licenciamiento por usuario y su enfoque hacia grandes organizaciones lo hace menos accesible para pequeñas y medianas empresas. Esta limitación no se da en la aplicación desarrollada ofreciendo una solución propia que pudiera adaptarse a diferentes tamaños de organización sin restricciones de licenciamiento.

\section{Enfoques académicos y de investigación}\label{enfoques-academicos-investigacion}
\subsection{Optimización de recursos compartidos}\label{optimizacion-recursos-compartidos}
En el ámbito académico, se han desarrollado investigaciones sobre algoritmos de optimización que buscan asignar de forma eficiente espacios compartidos, aprovechando patrones de uso histórico y predicciones de demanda. Entre estos enfoques destaca la idea de la ''reserva inteligente'', capaz de sugerir horarios y espacios de manera automática según la disponibilidad de los participantes y el historial de uso.
Aunque se trata de soluciones técnicamente avanzadas, su puesta en práctica requiere disponer de grandes volúmenes de datos y capacidades de \emph{machine learning} que superan el alcance de sistemas más sencillos. Aun así, los principios que persiguen optimizar la experiencia del usuario y reducir conflictos han servido como base para diseñar las funciones de validación incluidas en este proyecto.

\subsection{Arquitecturas de sistemas colaborativos}\label{arquitecturas-sistemas-colaborativos}
La investigación sobre arquitecturas escalables para sistemas de colaboración empresarial ofrece una base sólida para diseñar aplicaciones web capaces de atender a muchos usuarios al mismo tiempo. Enfoques como la arquitectura de microservicios o el uso de patrones como CQRS (Command Query Responsibility Segregation) aportan ideas útiles para gestionar la consistencia de los datos en sistemas de reservas.
Si bien las arquitecturas distribuidas pueden ser más complejas de lo que un sistema básico requiere, sus principios —como la separación de responsabilidades y la gestión eficiente de la concurrencia— han inspirado la elección de una arquitectura por capas, con una clara división entre las operaciones de lectura y escritura.

\subsection{Estudios de usabilidad en sistemas empresariales}\label{estudios-usuabilidad-sistemas-empresariales}
En los sistemas de gestión empresarial, la experiencia del usuario es clave para que las personas realmente adopten la herramienta y se sientan cómodas usándola. Factores como que el proceso de reserva sea sencillo, que la disponibilidad se muestre de forma clara y que sea fácil hacer cambios sin perderse, pueden marcar la diferencia.
Teniendo esto en cuenta, en este proyecto se han incorporado mejoras como calendarios visuales para ver la disponibilidad de un vistazo, formularios que dan respuesta inmediata a las acciones del usuario y flujos de trabajo más simples para las tareas que se realizan con mayor frecuencia.

\section{Frameworks y metodologías de desarrollo}\label{frameworks-metodologias-desarrollo}
\subsection{Spring Boot en aplicaciones empresariales}\label{spring-boot-aplicaciones-empresariales}
La adopción de Spring Boot en el desarrollo de aplicaciones web empresariales ha demostrado ventajas significativas en términos de productividad y mantenibilidad. Habiendo analizado diferentes modelos arquitectónicos, se muestra que la combinación de Spring Boot con arquitecturas por capas resulta en sistemas más robustos y fáciles de mantener.
Estas conclusiones han servido de apoyo para las decisiones arquitectónicas adoptadas en el presente TFG, especialmente en cuanto al uso de patrones como \emph{Repository}, \emph{Service Layer} y la integración con Spring Security para la gestión de autenticación y autorización.

\subsection{Metodologías ágiles en contextos académicos}\label{metodologias-agiles-contextos-academicos}
La aplicación de metodologías ágiles en el contexto específico de proyectos académicos ha demostrado beneficios en la gestión de restricciones temporales y requisitos evolutivos. Los enfoques de ''Scrum Académico'' adaptan los principios ágiles a las restricciones específicas de proyectos educativos, incluyendo la gestión de revisiones con los tutores y la documentación académica requerida.
Esta metodología ha sido influyente en la planificación del presente proyecto, especialmente en la definición de \emph{sprints} alineados con las revisiones académicas y la gestión del \emph{backlog} considerando tanto objetivos funcionales como requisitos de documentación.

\section{Posicionamiento del proyecto desarrollado}\label{posicionamiento-proyecto-desarrollado}
Frente a las soluciones analizadas, el presente proyecto se posiciona como una alternativa que combina las siguientes características:

\begin{itemize}
\tightlist
\item
\textbf{Control y flexibilidad}: Al ser una solución de código abierto, permite personalización completa sin restricciones comerciales.
\item
\textbf{Arquitectura moderna}: Utiliza tecnologías actuales y patrones de diseño consolidados.
\item
\textbf{Enfoque híbrido}: Combina simplicidad de uso con funcionalidades avanzadas según sea necesario.
\item
\textbf{Escalabilidad progresiva}: Diseño que permite evolución gradual según crecen las necesidades
\item
\textbf{Sistema híbrido de duración}: Flexibilidad para manejar tanto reservas libres como slots predefinidos.
\item
\textbf{Gestión colaborativa}: Funcionalidades de convocatoria con notificaciones integradas.
\item
\textbf{Arquitectura educativa}: Implementación que sirve como caso de estudio para futuros desarrollos.
\item
\textbf{Metodología adaptada}: Aplicación práctica de Scrum en contexto académico.
\end{itemize}

\subsection{Limitaciones y contexto}\label{limitaciones-contexto}
Es importante reconocer las limitaciones del proyecto actual:

\begin{itemize}
\tightlist
\item
\textbf{Alcance funcional}: Enfocado en necesidades básicas de gestión de reservas.
\item
\textbf{Funcionalidades avanzadas}: Ausencia de características como IA predictiva o análisis avanzado.
\item
\textbf{Integración}: Limitaciones en conectividad con sistemas empresariales complejos.
\end{itemize}
