\apendice{Especificación de diseño}

\section{Introducción}

En esta sección se detalla la especificación de diseño de la aplicación ``ReservApp''. El objetivo de este documento es proporcionar una visión clara y detallada de la arquitectura de software, el diseño de la base de datos y los flujos de procedimiento de las funcionalidades principales de la aplicación.

``ReservApp'' es una aplicación web desarrollada con el framework Spring Boot, diseñada para la gestión de reservas en diversos tipos de establecimientos. Permite a los usuarios registrarse, buscar establecimientos, realizar reservas y gestionar convocatorias asociadas a dichas reservas.

Este documento está estructurado en las siguientes secciones principales:
\begin{itemize}
    \item \textbf{Diseño de datos:} Describe el modelo de datos de la aplicación, incluyendo las entidades, sus atributos y las relaciones entre ellas. Se adjuntan los diagramas Entidad-Relación y Relacional.
    \item \textbf{Diseño arquitectónico:} Explica la arquitectura de software sobre la que se construye la aplicación, detallando las capas, los patrones de diseño empleados y las tecnologías principales.
    \item \textbf{Diseño procedimental:} Ilustra la secuencia de operaciones para los casos de uso más importantes de la aplicación mediante diagramas de secuencia.
\end{itemize}

El sistema está diseñado siguiendo principios de arquitectura en capas, separación de responsabilidades y buenas prácticas de desarrollo con Spring Framework.

El propósito de esta memoria es servir como guía técnica para un equipo de desarrollo y mantenimiento, facilitando la comprensión del sistema y sirviendo como base para futuras evoluciones del mismo.

\section{Diseño de datos}

El modelo de datos de la aplicación ``ReservApp'' está compuesto por un conjunto de entidades que representan los conceptos del dominio del problema. A continuación se describen las entidades principales, sus atributos y las relaciones que existen entre ellas.

\subsection{Descripción de Entidades}

\subsubsection{Usuario}
Representa a una persona que utiliza la aplicación. Puede ser un cliente que realiza reservas o un administrador del sistema.
\begin{itemize}
    \item \textbf{id (PK):} Identificador único del usuario (String).
    \item \textbf{nombre:} Nombre del usuario.
    \item \textbf{apellidos:} Apellidos del usuario.
    \item \textbf{correo:} Dirección de correo electrónico (única).
    \item \textbf{password:} Contraseña para la autenticación.
    \item \textbf{telefono:} Número de teléfono de contacto.
    \item \textbf{administrador:} Flag booleano que indica si el usuario tiene privilegios de administrador.
    \item \textbf{bloqueado:} Flag booleano para restringir el acceso al usuario.
\end{itemize}

\subsubsection{Establecimiento}
Representa un lugar físico o recurso que puede ser reservado por los usuarios.
\begin{itemize}
    \item \textbf{id (PK):} Identificador numérico autoincremental.
    \item \textbf{nombre:} Nombre del establecimiento.
    \item \textbf{descripcion:} Descripción detallada del establecimiento.
    \item \textbf{aforo:} Determina el número de reservas simultáneas que se pueden realizar.
    \item \textbf{duracionReserva:} Duración estándar en minutos de una reserva.
    \item \textbf{capacidad:} Capacidad máxima de personas por reserva.
    \item \textbf{activo:} Flag booleano que indica si el establecimiento está disponible para nuevas reservas.
\end{itemize}

\subsubsection{FranjaHoraria}
Define los períodos de tiempo en los que un establecimiento está abierto y disponible para ser reservado.
\begin{itemize}
    \item \textbf{id (PK):} Identificador numérico autoincremental.
    \item \textbf{diaSemana:} Día de la semana (Lunes, Martes, etc.).
    \item \textbf{horaInicio:} Hora de inicio de la franja.
    \item \textbf{horaFin:} Hora de fin de la franja.
\end{itemize}

\subsubsection{Reserva}
Representa una reserva realizada por un usuario en un establecimiento para una fecha y hora específicas.
\begin{itemize}
    \item \textbf{id (PK):} Identificador numérico autoincremental.
    \item \textbf{fechaReserva:} Fecha y hora de inicio de la reserva.
    \item \textbf{horaFin:} Hora de finalización de la reserva.
\end{itemize}

\subsubsection{Convocatoria}
Representa una invitación o evento asociado a una reserva, permitiendo registrar a varios participantes.
\begin{itemize}
    \item \textbf{id (PK):} Identificador de la reserva asociada (clave foránea y primaria).
    \item \textbf{enlace:} Un enlace externo asociado a la convocatoria (ej. para una videollamada).
    \item \textbf{observaciones:} Notas o comentarios adicionales.
\end{itemize}

\subsubsection{Convocado}
Entidad intermedia que asocia un `Usuario` con una `Convocatoria`.
\begin{itemize}
    \item \textbf{id (PK):} Clave primaria compuesta por `idReserva` y `idUsuario`.
\end{itemize}

\subsection{Relaciones entre Entidades}
\begin{itemize}
    \item Un \textbf{Usuario} puede realiza múltiples \textbf{Reservas} (1:0..N).
    \item Un \textbf{Establecimiento} puede tener múltiples \textbf{Reservas} (1:0..N).
    \item Un \textbf{Establecimiento} tiene una o varias \textbf{Franjas Horarias} (1:1..N).
    \item Una \textbf{Reserva} está asociada a un único \textbf{Usuario} y un único \textbf{Establecimiento}.
    \item Una \textbf{Reserva} puede tener una única \textbf{Convocatoria} (1:0..1).
    \item Una \textbf{Convocatoria} puede agrupar a múltiples \textbf{Usuarios} a través de la entidad \textbf{Convocado} (1:0..N).
\end{itemize}



\section{Diseño arquitectónico}

\section{Diseño procedimental}



