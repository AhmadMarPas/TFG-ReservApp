\apendice{Plan de Proyecto Software}

\section{Introducción}
En el presente plan de proyecto se define la estrategia de desarrollo mediante la cual se ha llevado a cabo la implementación de un sistema web de gestión de reservas de salas de reuniones. Este anexo establece el marco organizativo y metodológico que se llevó a cabo durante el proceso de desarrollo, desde la concepción inicial hasta la puesta en producción del sistema. La planificación aquí presentada contempla una estimación temporal realista de las diferentes fases del proyecto, considerando las actividades de análisis, diseño, implementación, pruebas y despliegue. Asimismo, se evalúa la viabilidad del proyecto desde una perspectiva económica y legal, analizando los costes asociados al desarrollo, los recursos necesarios y el cumplimiento de la normativa aplicable en materia de protección de datos y propiedad intelectual.

\section{Planificación temporal}
Para el desarrollo de la aplicación de gestión de reservas de salas de reuniones se ha adoptado la metodología ágil Scrum como marco de trabajo principal. Esto ha sido debido a que Scrum, por la naturaleza iterativa e incremental, permite que el proyecto se adapte en cualquier momento a los requisitos cambiantes y una entrega de valor constante a lo largo del proceso de desarrollo.

Scrum proporciona un marco estructurado que facilita la organización del trabajo en iteraciones cortas llamadas sprints, que normalmente suelen ser de 2 semanas de duración, pero por las condiciones del actual proyecto se han establecido en una sola semana. Scrum permite una mayor flexibilidad en la gestión de cambios, una mejor comunicación con los tutores y una entrega más predecible de funcionalidades. Además, la transparencia inherente a Scrum facilita la identificación temprana de impedimentos y riesgos, permitiendo su resolución antes de que impacten significativamente en el proyecto.

La gestión y seguimiento de los sprints se ha realizado utilizando Zube~\cite{zube}, que es una plataforma especializada en la gestión de proyectos ágiles y que se integra perfectamente con el repositorio de GitHub. Además, Zube ofrece funcionalidades específicas para la implementación de Scrum, que incluyen:

\begin{itemize}
\tightlist
\item
Tableros Kanban personalizables: Permiten visualizar el flujo de trabajo desde el backlog hasta la finalización de las tareas.
\item
Gestión de sprints: Dispone de herramientas para la planificación, seguimiento y retrospectivas de cada iteración.
\item
Métricas de rendimiento: Incluyen gráficos de \emph{burndown charts}, \emph{velocity tracking} y análisis de ciclo de vida de las tareas.
\item
Integración nativa con GitHub: Permite la sincronización automática con \emph{issues}, \emph{pull requests} y \emph{commits}.
\end{itemize}

\section{Estudio de viabilidad}


\subsection{Viabilidad económica}


\subsection{Viabilidad legal}


