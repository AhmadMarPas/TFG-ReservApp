\apendice{Plan de Proyecto Software}

\section{Introducción}
En el presente plan de proyecto software se define la estrategia de desarrollo que se ha llevado a cabo para la implementación de un sistema web de gestión de reservas de salas de reuniones. Este anexo establece el marco organizativo y metodológico que se llevó a cabo durante el proceso de desarrollo, desde la concepción inicial hasta la puesta en producción del sistema. La planificación que aquí se presenta contempla una estimación temporal realista de las diferentes fases del proyecto, considerando las actividades de análisis, diseño, implementación, pruebas y despliegue. Asimismo, se evalúa la viabilidad del proyecto desde una perspectiva económica y legal, analizando los costes asociados al desarrollo, los recursos necesarios y el cumplimiento de la normativa aplicable en materia de protección de datos y propiedad intelectual.

\section{Planificación temporal}
Para el desarrollo de la aplicación de gestión de reservas de salas de reuniones se ha adoptado la metodología ágil Scrum como marco de trabajo principal. Esta elección se basa en la naturaleza iterativa e incremental que requiere el proyecto, permitiendo una adaptación continua a los requisitos cambiantes y una entrega de valor constante a lo largo del proceso de desarrollo.

Scrum proporciona un marco estructurado que facilita la organización del trabajo en iteraciones cortas llamadas sprints, que se realizan normalmente en un periodo de 2 semanas de duración. Scrum permite una mayor flexibilidad en la gestión de cambios, una mejor comunicación con los tutores y una entrega más predecible de funcionalidades. Además, la transparencia inherente a Scrum facilita la identificación temprana de impedimentos y riesgos, permitiendo su resolución antes de que impacten significativamente en el proyecto.

La gestión y seguimiento de los sprints se ha realizado utilizando Zube~\cite{zube}, que es una plataforma especializada en la gestión de proyectos ágiles y que se integra perfectamente con el repositorio de GitHub. Además, Zube ofrece funcionalidades específicas para la implementación de Scrum, que incluyen:

\begin{itemize}
\tightlist
\item
\textbf{Tableros Kanban personalizables}: Permiten visualizar el flujo de trabajo desde el backlog hasta la finalización de las tareas. Como ejemplo de ello se puede observa la figura ~\ref{fig:Zube-Kanban}.

\imagen{Zube-Kanban}{Ejemplo de las tarjetas de un sprint en Zube con los diferentes estados en los que se puede encontrar.}

\item
\textbf{Gestión de sprints}: Dispone de herramientas para la planificación, seguimiento y retrospectivas de cada iteración.
\item
\textbf{Métricas de rendimiento}: Incluyen gráficos de \emph{burndown charts}, \emph{velocity tracking} y análisis de ciclo de vida de las tareas. En la figura ~\ref{fig:Zube-Burndown} se puede observar un ejemplo de gráfica de \emph{burndown}.

\imagen{Zube-Burndown}{Ejemplo de gráfico de \emph{burndown} en el que se muestra la evolución de los puntos realizados.}

\item
\textbf{Integración nativa con GitHub}: Permite la sincronización automática con \emph{issues}, \emph{pull requests} y \emph{commits}.
\end{itemize}

La elección de Zube se debe a su capacidad de proporcionar una visión global del progreso del proyecto, facilitando con ello tanto la planificación a corto plazo como el seguimiento de objetivos a largo plazo. Su interfaz es intuitiva y permite una gestión eficiente del \emph{product backlog} y la asignación de tareas a cada sprint.

Complementariamente al uso de Zube, se ha utilizado el sistema de \emph{issues} de GitHub para el registro detallado de todas las tareas, errores, mejoras y funcionalidades del proyecto. Esta integración dual proporciona varios beneficios:

\begin{itemize}
\tightlist
\item
\textbf{Trazabilidad completa}: Cada \emph{issue} está vinculada directamente con el código fuente, \emph{commits} y \emph{pull requests} correspondientes. En la figura ~\ref{fig:GitHub-Issue3} se puede observar un ejemplo del histórico de una \emph{issue}.
\item
\textbf{Documentación técnica}: Las \emph{issues} permiten documentar decisiones técnicas, problemas encontrados y soluciones implementadas.
\item
\textbf{Colaboración eficiente}: Facilita la comunicación sobre aspectos técnicos específicos y la revisión de código.
\item
\textbf{Historial detallado}: Mantiene un registro completo de la evolución del proyecto y las decisiones tomadas.
\end{itemize}

\imagen{GitHub-Issue3}{Ejemplo del histórico de una \emph{issue}.}

\subsection{Estructura y Duración del Proyecto}
El proyecto se ha estructurado en 11 sprints de 2 semanas cada uno, resultando en una duración total de 22 semanas (aproximadamente 5 meses). Esta planificación se ha diseñado considerando la complejidad del sistema a desarrollar y los recursos disponibles, permitiendo un desarrollo incremental y sostenible.

Cabe señalar que la distribución temporal del proyecto no ha seguido un patrón estrictamente lineal debido a interrupciones planificadas relacionadas con el calendario académico que ha requerido pausas en el desarrollo del proyecto, adaptando la planificación a las exigencias del curso académico. Esta variabilidad temporal ha sido gestionada mediante la flexibilidad que ofrece la metodología Scrum, permitiendo ajustar la duración de algunos sprints y redistribuir las cargas de trabajo según la disponibilidad. A pesar de estas interrupciones, se ha mantenido la estructura de 11 sprints planificada, asegurando que los objetivos de cada iteración se cumplieran independientemente de las fluctuaciones en el ritmo de trabajo, quédando resumida la planificación en los siguientes puntos:

\begin{itemize}
\tightlist
\item
\textbf{Duración total del proyecto}: 22 semanas.
\item
\textbf{Número de sprints}: 11.
\item
\textbf{Duración de cada sprint}: 2 semanas.
\item
\textbf{Eventos Scrum}: Planificación de sprint, \emph{daily standups}, revisión y retrospectiva.
\item
\textbf{Período de desarrollo activo}: 19 semanas.
\item
\textbf{Período de cierre y documentación}: 3 semanas.
\end{itemize}

\subsection{Detalle de los Sprints}

\subsubsection{Sprint 0 - Configuración Inicial y Arquitectura Base (17/03/2025 – 30/03/2025)} 
En este sprint inicial se establecieron los cimientos del proyecto ReservApp como aplicación web para gestión de reservas. Se configuró el entorno de desarrollo con Spring Boot como framework principal, estableciendo una arquitectura basada en capas (presentación, servicio, persistencia). Se configuró la estructura Maven con gestión de dependencias, incluyendo Spring Data JPA para persistencia, Spring Security para autenticación, Thymeleaf como motor de templates y MySQL como base de datos principal. En esta configuración inicial se priorizó la escalabilidad y mantenibilidad, preparando el proyecto para futuras ampliaciones mediante Docker Compose y herramientas de calidad continua como SonarCloud y JaCoCo para análisis de cobertura de código.

\textbf{Objetivos principales}:
\begin{itemize}
\tightlist
\item
Configurar proyecto Spring Boot con Maven.
\item
Establecer arquitectura en capas (controller, service, repository, model).
\item
Configurar la base de datos MySQL con pool de conexiones.
\item
Implementar configuración de desarrollo con Docker Compose.
\item
Establecer herramientas de calidad de código (SonarCloud, JaCoCo).
\end{itemize}

\textbf{Actividades realizadas}:
\begin{itemize}
\tightlist
\item
Creación del 'pom.xml' con todas las dependencias necesarias.
\item
Configuración de `application.properties` con parámetros de BD y logging.
\item
Estructura de paquetes modular: 'controller', 'service', 'model', 'config'.
\item
Configuración de 'compose.yaml' para contenedores de desarrollo.
\item
Integración con SonarCloud mediante 'sonar-project.properties' para análisis continuo.
\item
Implementación de 'ReservApplication' como clase principal con configuración de componentes
\end{itemize}


\subsubsection{Sprint 1 - Modelo de Datos y Entidades Base (31/03/2025 – 13/04/2025)} 
En este sprint se diseñó e implementó el modelo de datos completo del sistema, creando las entidades principales que sustentan la funcionalidad de reservas. Se desarrolló una jerarquía de entidades con 'EntidadInfo' como clase base que proporciona auditoría automática, y se implementaron las entidades principales: 'Usuario', 'Establecimiento', 'Reserva', 'Perfil' y 'Menu'. Se establecieron las relaciones entre entidades usando JPA annotations, implementando validaciones con Bean Validation y configurando índices de base de datos para optimizar las consultas más frecuentes. También se creó el sistema de claves primarias personalizadas con 'EntidadPK' y 'EntidadID'.

\textbf{Objetivos principales}:
\begin{itemize}
\tightlist
\item
Diseñar modelo de datos completo del sistema.
\item
Implementar entidades JPA con validaciones.
\item
Establecer relaciones entre entidades.
\item
Configurar sistema de auditoría automática con interceptores de entidad personalizados.
\item
Optimizar consultas con índices de base de datos.
\end{itemize}

\textbf{Actividades realizadas}:
\begin{itemize}
\tightlist
\item
Creación de 'EntidadInfo' como clase base con auditoría.
\item
Implementación de entidades: 'Usuario', 'Establecimiento', 'Reserva'.
\item
Desarrollo de 'Perfil' y 'Menu' para gestión de perfiles.
\item
Configuración de relaciones '@OneToMany', '@ManyToMany', '@ManyToOne'.
\item
Implementación de validaciones con '@NotNull', '@Email', '@Pattern'.
\end{itemize}

\subsubsection{Sprint 2 - Seguridad y Autenticación (14/04/2025 – 27/04/2025)} 
Este sprint se centró en implementar un sistema de seguridad robusto utilizando Spring Security y, para ello, se desarrolló un sistema de autenticación personalizado con 'CustomUserDetailsService' y 'CustomAuthenticationSuccessHandler', implementando encriptación de contraseñas con BCrypt y control de acceso basado en roles (USER/ADMIN). Se configuró la seguridad web con páginas de login personalizadas, protección contra ataques comunes (CSRF, \emph{session fixation}, \emph{clickjacking}) y manejo de sesiones. También se implementó el sistema de registro de usuarios con validación de email y tokens de confirmación, y se estableció un sistema de gestión de sesiones con timeout configurable y logout seguro.

\textbf{Objetivos principales}:
\begin{itemize}
\tightlist
\item
Implementar autenticación con Spring Security.
\item
Crear sistema de roles y permisos.
\item
Desarrollar páginas de login y registro personalizadas.
\item
Configurar encriptación de contraseñas.
\item
Implementar control de sesiones y logout
\end{itemize}

\textbf{Actividades realizadas}:
\begin{itemize}
\tightlist
\item
Desarrollo de 'SecurityConfig' con configuración completa.
\item
Creación de 'CustomUserDetailsService' para carga de usuarios.
\item
Implementación de 'CustomAuthenticationSuccessHandler'.
\item
Desarrollo de 'LoginController' con registro y autenticación.
\item
Creación de templates 'login.html' y 'registro.html'
\end{itemize}

\subsubsection{Sprint 3 - Gestión de Usuarios y Perfiles (28/04/2025 – 11/05/2025)} 
En este sprint se desarrolló la funcionalidad completa de gestión de usuarios y perfiles del sistema. Se implementaron los servicios 'UsuarioService' y 'PerfilService' con operaciones CRUD completas, incluyendo funcionalidades avanzadas como búsqueda de usuarios, asignación de establecimientos y gestión de perfiles personalizados. Se creó el panel de administración para gestión de usuarios con capacidades de bloqueo/desbloqueo, edición de perfiles y asignación de roles. También se implementó la funcionalidad de edición de perfil personal y la gestión de menús asociados a perfiles, estableciendo la base para un sistema de permisos granular.

\textbf{Objetivos principales}:
\begin{itemize}
\tightlist
\item
Desarrollar CRUD completo para usuarios.
\item
Implementar gestión de perfiles personalizados.
\item
Crear panel de administración de usuarios.
\item
Desarrollar funcionalidad de búsqueda y filtrado.
\item
Implementar asignación de establecimientos a usuarios
\end{itemize}

\textbf{Actividades realizadas}:
\begin{itemize}
\tightlist
\item
Implementación de 'UsuarioServiceImpl' con operaciones completas.
\item
Desarrollo de 'PerfilServiceImpl' y gestión de menús.
\item
Creación de 'UserManagementController' para administración.
\item
Desarrollo de 'EditarPerfilController' para usuarios.
\item
Implementación de templates de gestión de usuarios y perfiles
\end{itemize}

\subsubsection{Sprint 4 - Gestión de Establecimientos (09/06/2025 – 22/06/2025)} 
Este sprint se enfocó en desarrollar el módulo completo de gestión de establecimientos, implementando funcionalidades para crear, editar y administrar los espacios disponibles para reservas. Se desarrolló el sistema de franjas horarias con 'FranjaHoraria' que permite definir horarios de apertura por día de la semana, y se implementó la lógica de aforo para controlar la capacidad máxima de reservas simultáneas. Se creó la funcionalidad de asignación de establecimientos a usuarios, permitiendo que cada usuario tenga acceso solo a los establecimientos autorizados. También se implementaron validaciones de negocio para asegurar la consistencia de los datos de establecimientos.

\textbf{Objetivos principales}:
\begin{itemize}
\tightlist
\item
Desarrollar CRUD completo para establecimientos.
\item
Implementar sistema de franjas horarias.
\item
Crear funcionalidad de aforo y capacidad.
\item
Desarrollar asignación de usuarios a establecimientos.
\item
Implementar validaciones de negocio específicas
\end{itemize}

\textbf{Actividades realizadas}:
\begin{itemize}
\tightlist
\item
Implementación de 'EstablecimientoServiceImpl' con lógica completa.
\item
Desarrollo de 'EstablecimientoController' y 'EstablecimientoAsignacionController'.
\item
Creación de entidad 'FranjaHoraria' con relaciones.
\item
Implementación de templates para gestión de establecimientos.
\item
Desarrollo de funcionalidad de asignación usuario-establecimiento.
\end{itemize}

\subsubsection{Sprint 5 - Sistema de Reservas Core (23/06/2025 – 06/07/2025)} 
En este sprint se desarrolló el núcleo del sistema de reservas, implementando la lógica completa para crear, modificar y gestionar reservas. Se creó 'ReservaService' con algoritmos sofisticados para verificar disponibilidad considerando aforo, franjas horarias y conflictos de tiempo. Se implementó el sistema de slots de tiempo con 'SlotReservaUtil' que genera automáticamente los horarios disponibles basados en la configuración del establecimiento. Se desarrolló la funcionalidad de calendario interactivo que permite a los usuarios visualizar y seleccionar fechas y horarios disponibles. También se implementó la lógica de validación de reservas que considera múltiples factores como horarios de apertura, aforo máximo y reservas existentes.

\textbf{Objetivos principales}:
\begin{itemize}
\tightlist
\item
Desarrollar lógica core de reservas.
\item
Implementar algoritmos de verificación de disponibilidad.
\item
Crear sistema de slots de intervalo de tiempo.
\item
Desarrollar calendario interactivo de reservas.
\item
Implementar validaciones complejas de negocio.
\end{itemize}

\textbf{Actividades realizadas}:
\begin{itemize}
\tightlist
\item
Implementación de 'ReservaServiceImpl' con lógica avanzada.
\item
Desarrollo de 'SlotReservaUtil' para generación de slots.
\item
Creación de algoritmos de verificación de aforo.
\item
Implementación de 'ReservaController' con validaciones.
\item
Desarrollo de templates de calendario y gestión de reservas.
\end{itemize}

\subsubsection{Sprint 6 - Sistema de Convocatorias (07/07/2025 – 20/07/2025)} 
Después de una revisión realizada con el tutor, surgió una nueva necesidad que fue incorporada y contemplada en este sprint. Se decidió desarrollar un sistema de convocatorias que permite invitar a otros usuarios a las reservas creadas. Se implementaron las entidades 'Convocatoria' y 'Convocado' con una relación compleja que maneja claves primarias compuestas mediante 'ConvocadoPK'. Se desarrolló la funcionalidad de búsqueda de usuarios para invitaciones, implementando un sistema AJAX que permite buscar y seleccionar usuarios dinámicamente. Se creó el sistema de notificaciones por email que informa a los usuarios convocados sobre las reservas, incluyendo detalles como fecha, hora, establecimiento y enlaces de reunión. También se implementó la gestión de convocatorias con capacidades de edición y eliminación.

\textbf{Objetivos principales}:
\begin{itemize}
\tightlist
\item
Desarrollar sistema completo de convocatorias.
\item
Implementar búsqueda dinámica de usuarios.
\item
Crear sistema de notificaciones por email.
\item
Desarrollar gestión de usuarios convocados.
\item
Implementar funcionalidad de enlaces de reunión.
\end{itemize}

\textbf{Actividades realizadas}:
\begin{itemize}
\tightlist
\item
Implementación de entidades 'Convocatoria' y 'Convocado'.
\item
Desarrollo de 'ConvocatoriaService' con lógica compleja.
\item
Creación de sistema de búsqueda AJAX de usuarios.
\item
Implementación de 'EmailService' para notificaciones.
\item
Desarrollo de funcionalidades de gestión de convocatorias.
\end{itemize}

\subsubsection{Sprint 7 - Funcionalidades Avanzadas y Optimizaciones (04/08/2025 – 10/08/2025)} 
Este sprint se enfocó en implementar funcionalidades avanzadas y optimizaciones de rendimiento identificadas durante las pruebas. Se desarrolló el sistema de paginación para listas grandes de datos, implementando scroll infinito en las reservas del usuario. Se creó funcionalidad AJAX para obtener slots disponibles dinámicamente y se implementaron optimizaciones de consultas para evitar problemas N+1. Se desarrolló el sistema de edición y anulación de reservas con validaciones que consideran fechas pasadas y permisos de usuario. También se implementaron mejoras en la gestión de convocatorias con \emph{soft delete}.

\textbf{Objetivos principales}:
\begin{itemize}
\tightlist
\item
Implementar funcionalidades avanzadas de reservas.
\item
Crear sistema de paginación y scroll infinito.
\item
Implementar funcionalidades AJAX dinámicas.
\item
Desarrollar validaciones de fechas de reservas.
\end{itemize}

\textbf{Actividades realizadas}:
\begin{itemize}
\tightlist
\item
Implementación de edición y anulación de reservas.
\item
Desarrollo de \emph{endpoints} AJAX para slots disponibles.
\item
Creación de sistema de paginación de reservas.
\item
Implementación de optimizaciones de consultas.
\item
Desarrollo de validaciones avanzadas de reservas.
\end{itemize}

\subsubsection{Sprint 8 - Testing, Documentación y Despliegue (11/08/2025 – 17/08/2025)} 
En este sprint se finalizó un conjunto amplio de tests unitarios y de integración que cubre todas las funcionalidades del sistema. Se implementaron tests para controladores, servicios, repositorios y entidades, alcanzando una cobertura de código significativa medida con JaCoCo. Se configuró el pipeline de CI/CD con integración a SonarCloud para análisis continuo de calidad de código. También se preparó la configuración de despliegue con perfiles específicos para desarrollo, testing y producción.

\textbf{Objetivos principales}:
\begin{itemize}
\tightlist
\item
Desarrollar suite completa de tests automatizados.
\item
Crear documentación técnica.
\item
Configurar pipeline de CI/CD con SonarCloud.
\item
Preparar configuración de despliegue multi-entorno.
\item
Realizar análisis final de calidad y rendimiento.
\end{itemize}

\textbf{Actividades realizadas}:
\begin{itemize}
\tightlist
\item
Implementación de tests unitarios para todos los componentes.
\item
Desarrollo de tests de integración con base de datos H2.
\item
Creación de la documentación técnica.
\item
Configuración de perfiles de aplicación para diferentes entornos.
\item
Preparación de configuración de despliegue y monitoreo.
\end{itemize}

\subsubsection{Sprint 9 - Despliegue en Render~\cite{render} con Base de Datos en Aiven~\cite{aiven} (18/08/2025 – 24/08/2025)} 
En el sprint final se investigaron diferentes opciones para el hosting de la aplicación y entre las diferentes opciones consultadas, finalmente se decantó por la ofrecida por Render ya que ha resultado ser la única gratuita.

\textbf{Objetivos principales}:
\begin{itemize}
\tightlist
\item
Configuración de hosting en Render para la aplicación web.
\item
Configuración de base de datos MySQL en Aiven.
\item
Establecimiento de conexión segura entre ambos servicios.
\item
Configuración de variables de entorno y secretos para producción.
\end{itemize}

\textbf{Actividades realizadas}:
\begin{itemize}
\tightlist
\item
Configuración de cuenta y proyecto en Render para hosting de la aplicación.
\item
Setup de base de datos MySQL administrada en Aiven con configuraciones de producción.
\item
Configuración de variables de entorno para conexión a base de datos externa.
\item
Implementación de SSL/TLS para conexiones seguras.
\item
Testing de conectividad y rendimiento en entorno de producción.
\end{itemize}

\subsubsection{Sprint 10 - Despliegue y Documentación Final (01/09/2025 – 14/09/2025)} 
En este sprint final se deja definitivamente la aplicación desplegada en el entorno de producción y se finaliza por completar la memoria y anexos correspondientes a la documentación del TFG.

\textbf{Objetivos principales}:
\begin{itemize}
\tightlist
\item
Despliegue en entorno de producción.
\item
Completar documentación técnica y de usuario.
\item
Configuración de monitoreo y logging en producción.
\item
Preparación de materiales de entrega.
\end{itemize}

\textbf{Actividades realizadas}:
\begin{itemize}
\tightlist
\item
Configuración y despliegue en servidor de producción.
\item
Configuración de monitoreo, alertas y backup.
\item
Creación de documentación completa de usuario y administrador.
\item
Preparación de manual de instalación y mantenimiento.
\item
Validación final del sistema en producción.
\end{itemize}







\section{Estudio de viabilidad}


\subsection{Viabilidad económica}


\subsection{Viabilidad legal}


