\apendice{Anexo de sostenibilización curricular}

\section{Introducción}
Este documento es una reflexión sobre cómo la sostenibilidad se integró en el proyecto ``ReservApp'', mi trabajo de fin de grado. Siguiendo las recomendaciones de la CRUE~\cite{crue}, la ``sostenibilización curricular'' no solo trata temas del medio ambiente, sino que busca un enfoque completo que incluya aspectos sociales, económicos y ambientales.

Aquí explico cómo un proyecto tecnológico como ReservApp me ayudó a desarrollar y aplicar estas competencias. El objetivo es mostrar cómo mi trabajo contribuye a formar un profesional más consciente de los desafíos globales y alineado con los Objetivos de Desarrollo Sostenible (ODS) de la ONU.

A lo largo de las siguientes secciones, analizaré el proyecto desde tres ángulos: social, económico y ambiental. Mostraré cómo las decisiones de diseño, la arquitectura y las funciones de la aplicación se conectan con los principios de la sostenibilidad.

\subsection{Sostenibilidad Social}
La \textbf{sostenibilidad social} trata de cuidar a las personas, la equidad y la armonía en una comunidad. Aunque una app de reservas parezca algo simple, ReservApp tiene un impacto directo aquí porque gestiona cómo la gente accede a los recursos de un lugar, como una universidad.

Uno de los aspectos destacables en este sentido es que el proyecto promueve la \textbf{equidad y la transparencia}. Al ofrecer un sistema único para todos, asegura que el acceso a espacios como salas de estudio sea justo y sin favoritismos. Esto evita confusiones y conflictos, creando un ambiente más tranquilo y equitativo. Básicamente, todos tienen las mismas oportunidades, y eso es importante para que la comunidad funcione bien.

Además, desarrollar esta aplicación me dio la oportunidad de aprender a usar la tecnología para un fin social. No se trataba de construir algo por construir, sino de resolver un problema real. Aprendí a entender las necesidades de la gente, a diseñar pensando en el usuario y a ver cómo la tecnología puede influir positivamente en las interacciones humanas. Al hacer la vida más fácil y organizada para estudiantes y personal, la aplicación mejora sutilmente su bienestar y el ambiente general.

\subsection{Sostenibilidad Económica}
La \textbf{sostenibilidad económica} busca ser eficiente e inteligente para no malgastar recursos valiosos. En un proyecto de software como ReservApp, esto se ve de varias maneras.

Por un lado, la aplicación ayuda a optimizar el uso de los espacios físicos. La aplicación ayuda a usar de forma más eficiente los recursos físicos. Las infraestructuras como aulas, laboratorios y equipos son muy caras de comprar y mantener. Un sistema como ReservApp se asegura de que estos recursos se usen al máximo, evitando que queden vacíos por falta de organización. Al optimizar su uso, la institución puede ahorrar mucho dinero al no tener que invertir en nuevas instalaciones.

Además, el proyecto se construyó con tecnologías de código abierto (Open Source) como Java, el \emph{framework} Spring y la base de datos MySQL. Esta decisión eliminó por completo los costos de licencias de software, lo que representa un modelo de desarrollo sostenible. Al usar este tipo de tecnologías, no solo aprovechamos un ecosistema robusto mantenido por una comunidad global, sino que también nos aseguramos de que el proyecto sea viable a largo plazo sin depender de empresas privadas con modelos de negocio restrictivos. Y no menos importante, las habilidades que adquirí con estas herramientas son un valor añadido que me acompañará en mi carrera profesional.

\subsection{Sostenibilidad Ambiental}
Aunque la \textbf{sostenibilidad ambiental} en un proyecto de software no siempre es obvia, su impacto es muy real. Se trata de usar menos energía y recursos para reducir nuestra huella ecológica.

El principal beneficio ambiental de ReservApp es la \textbf{reducción del uso de recursos materiales}. Al digitalizar el proceso de reserva, se elimina la necesidad de sistemas basados en papel como registros y formularios. El impacto de una sola hoja es pequeño, pero en el contexto de una institución grande, el ahorro acumulado de papel, tinta y los recursos de producción es significativo.

Además, de manera indirecta, la optimización en la gestión de espacios también contribuye al medio ambiente. Un uso más eficiente de las salas permite una mejor gestión de los recursos energéticos. Por ejemplo, al conocer la ocupación de los espacios, se puede optimizar la climatización y la iluminación, evitando el derroche de energía en áreas que no están en uso.

Por último, el propio diseño del software es un factor de sostenibilidad. La elección de una arquitectura optimizada no solo mejora el rendimiento de la aplicación, sino que también reduce la carga de trabajo de los servidores. Esto se traduce en un menor consumo energético en los centros de datos, lo cual tiene un impacto positivo en el medio ambiente.

\subsection{Conclusión}
Con "ReservApp" descubrí que el desarrollo de software va mucho más allá de la codificación. Fue un proyecto que me enseñó a aplicar mis conocimientos para diseñar soluciones que consideren la sostenibilidad en todas sus facetas.

A través de este trabajo, aprendí a crear soluciones tecnológicas que no solo funcionan bien, sino que también son justas para las personas, viables económicamente y respetuosas con el medio ambiente. Creo que esta visión es esencial para afrontar los desafíos del siglo XXI y para que, como ingenieros de software, podamos aportar nuestro granito de arena a un futuro más sostenible.