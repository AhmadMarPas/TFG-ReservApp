\apendice{Especificación de Requisitos}

\section{Introducción}
Este documento presenta la especificación completa de requisitos para ReservApp, un sistema web de gestión de reservas. El objetivo es definir de manera precisa y detallada las funcionalidades, características técnicas y restricciones del sistema, proporcionando una base sólida para el desarrollo, testing y mantenimiento de la aplicación.

ReservApp es una aplicación web diseñada para facilitar la gestión integral de reservas en establecimientos diversos. El sistema permite a los usuarios realizar reservas en diferentes establecimientos, gestionar convocatorias con otros usuarios, y a los administradores supervisar todo el proceso de reservas y gestión de usuarios.

\section{Objetivos generales}

El objetivo principal del actual TFG es el desarrollo de un sistema web de gestión de reservas que permita a los usuarios realizar reservas de manera eficiente en diferentes establecimientos, con capacidades avanzadas de convocatorias y administración centralizada. Y como consecuencia de ello, se deben contemplar los siguientes objetivos específicos:

\begin{itemize}
\tightlist
\item
Gestión de Establecimientos.
    \begin{itemize}
    \tightlist
    \item
    Permitir la configuración flexible de establecimientos con horarios específicos.
    \item
    Implementar sistema de aforo para controlar la capacidad de reservas.
    \item
    Proporcionar asignación de usuarios a establecimientos específicos.
    \item
    Facilitar la gestión administrativa de establecimientos.
    \end{itemize}

\item
Gestión de Usuarios.
    \begin{itemize}
    \tightlist
    \item
    Proporcionar un sistema de registro y autenticación seguro.
    \item
    Implementar control de acceso basado en roles (Usuario/Administrador).
    \item
    Permitir la gestión de perfiles de usuario personalizables.
    \end{itemize}

\item
Sistema de Reservas.
    \begin{itemize}
    \tightlist
    \item
    Ofrecer múltiples modalidades de reserva (slots predefinidos y horarios libres).
    \item
    Implementar validación automática de disponibilidad y conflictos.
    \item
    Proporcionar calendario interactivo para visualización de reservas.
    \item
    Permitir modificación y anulación de reservas con notificaciones
    \end{itemize}

\item
Sistema de Convocatorias
    \begin{itemize}
    \tightlist
    \item
    Facilitar la invitación de otros usuarios a reservas.
    \item
    Proporcionar gestión de enlaces de reunión y observaciones.
    \item
    Implementar búsqueda avanzada de usuarios para convocatorias.
    \item
    Automatizar notificaciones por correo electrónico.
    \end{itemize}

\item
Administración del Sistema
    \begin{itemize}
    \tightlist
    \item
    Proporcionar un panel de administración completo.
    \item
    Implementar gestión avanzada de usuarios (bloqueo, desbloqueo, eliminación).
    \item
    Ofrecer visualización de reservas por establecimiento y fecha.
    \end{itemize}

\item
Experiencia de Usuario.
    \begin{itemize}
    \tightlist
    \item
    Desarrollar interfaz intuitiva y responsiva.
    \item
    Implementar navegación eficiente entre funcionalidades.
    \item
    Optimizar rendimiento para carga rápida de páginas.
    \item
    Proporcionar feedback visual inmediato de acciones
    \end{itemize}

\end{itemize}

\newpage
\section{Catálogo de requisitos}
En este apartado se enumeran los requisitos de la aplicación, organizándose éstos por tipo y se proporciona un identificador único (RF para Requisito Funcional, RNF para Requisito No Funcional) y una descripción del mismo.

\subsection{RF - Requisitos Funcionales}

\begin{itemize}
\tightlist
\item
\textbf{RF-001 - Gestión de Autenticación y Autorización}: El sistema debe proporcionar mecanismos seguros de autenticación y autorización basados en roles.
\item
\textbf{RF-002 - Registro de Usuarios}: El sistema debe permitir el registro de nuevos usuarios con validación de datos.
\item
\textbf{RF-003 - Gestión de Establecimientos}: El sistema debe permitir la gestión completa de establecimientos con configuración de horarios.
\item
\textbf{RF-004 - Sistema de Reservas}: El sistema debe permitir crear, modificar y anular reservas con validación de disponibilidad.
\item
\textbf{RF-005 - Sistema de Convocatorias}: El sistema debe permitir invitar a otros usuarios a reservas con información adicional.
\item
\textbf{RF-006 - Búsqueda de Usuarios}: El sistema debe proporcionar, para la configuración de convocatorias, una búsqueda eficiente de usuarios.
\item
\textbf{RF-007 - Administración de Usuarios}: El sistema debe permitir a los administradores gestionar cuentas de usuario.
\item
\textbf{RF-008 - Visualización de Reservas Administrativas}: El sistema debe proporcionar vista administrativa de reservas por establecimiento.
\item
\textbf{RF-009 - Gestión de Perfiles}: El sistema debe permitir la gestión de perfiles con menús asociados.
\item
\textbf{RF-010 - Notificaciones por Email}: El sistema debe enviar notificaciones por email para los diferentes eventos de reservas.
\end{itemize}

\subsection{RNF - Requisitos No Funcionales}
\begin{itemize}
\tightlist
\item
\textbf{RNF-001 - Rendimiento}: El sistema debe mantener tiempos de respuesta razonables de modo que la interacción o experiencia de usuario no se vea perjudicada. Los tiempos de carga y la respuesta de las operaciones deben ser aceptables.
\item
\textbf{RNF-002 - Seguridad}: El sistema debe implementar medidas de seguridad robustas, utilizando para ello un sistema de encriptación para las contraseñas, así como establecer una protección contra ataques. Del mismo modo, se debe establecer un control de acceso basado en roles y establecer un timeout de sesión automático.
\item
\textbf{RNF-003 - Usabilidad}: El sistema debe ser intuitiva y accesible, con un diseño adaptable a dispositivos móviles. La navegación principal estará optimizada para que el usuario acceda a las funciones principales en solo tres clics. Para una mejor interacción, se mostrarán mensajes de error claros y se ofrecerá un feedback visual instantáneo, asegurando que el diseño sea coherente en todas las páginas.

\item
\textbf{RNF-004 - Disponibilidad}: El sistema debe mantener una disponibilidad del 99\% durante el horario laboral. En caso de fallos inesperados, se establece un tiempo de recuperación inferior a 5 minutos, minimizando así el impacto en las operaciones. El mantenimiento preventivo incluye un backup automático diario para proteger la integridad de los datos. Además, el sistema manejará los errores de forma controlada y eficiente, asegurando que nunca se produzca pérdida de datos. Finalmente, se generarán logs detallados que facilitarán el diagnóstico y la resolución rápida de cualquier problema técnico.
\item
\textbf{RNF-005 - Escalabilidad}: El sistema debe ser capaz de poder absorver un crecimiento del número de usuarios, adaptándose sin problemas y ofreciendo un rendimiento adecuado incluso en situaciones de alta demanda.
\item
\textbf{RNF-006 - Mantenibilidad}: El sistema debe ser fácil de actualizar o modificar, siguiendo un diseño modular y una estructura clara que facilite su futuro mantenimiento. Se priorizará la legibilidad y la consistencia en la implementación para simplificar la corrección de errores y la incorporación de nuevas funcionalidades.
\item
\textbf{RNF-007 - Compatibilidad}: El sistema deberá ser compatible con los navegadores web más utilizados, incluyendo Google Chrome, Mozilla Firefox, Microsoft Edge y Safari. La interfaz de usuario será totalmente responsiva, garantizando que la aplicación se adapte y funcione correctamente en una amplia gama de dispositivos, como computadoras de escritorio, tabletas y teléfonos móviles, sin perder funcionalidad ni usabilidad.
\end{itemize}

\newpage
\section{Especificación de requisitos}

\subsection{Actores del sistema}

\begin{table}[H]
	\centering
	\begin{tabularx}{\linewidth}{ p{0.21\columnwidth} p{0.71\columnwidth} }
		\toprule
		\textbf{Actor}    & A01 \\
		\toprule
		\textbf{Nombre:} 			  & \textbf{Usuario} \\
		\textbf{Versión}              & 1.0    \\
		\textbf{Autor}                & \nombre \\
		\textbf{Descripción}          & Persona que interactúa con la aplicación ReservApp para las diferentes opciones que en ella se ofrecen. \\
		\textbf{Tipo}                 & Usuario \\
		\textbf{Objetivo}             & Crear, modificar o anular reservas, crear, modificar o anular convocatorias. \\
		\textbf{Responsabilidades}    & 
		\begin{itemize}
			\tightlist
			\item Crear, modificar y anular reservas.
			\item Gestionar convocatorias.
			\item Editar perfil personal.
		\end{itemize}\\
		\textbf{Relaciones con casos de uso} & CU01(\ref{cu:autenticacion-usuario}), CU02(\ref{cu:registro-usuario}), CU03(\ref{cu:crear-reserva}), CU04(\ref{cu:modificar-reserva}), CU05(\ref{cu:anular-reserva}), CU09(\ref{cu:logout}), CU10(\ref{cu:visualizar-reservas}), CU11(\ref{cu:buscar-usuarios}), CU12(\ref{cu:gestionar-convocatorias}). \\
		\bottomrule
	\end{tabularx}
	\caption{A01 - Usuario}
	\label{actor:usuario}
\end{table}


\begin{table}[H]
	\centering
	\begin{tabularx}{\linewidth}{ p{0.21\columnwidth} p{0.71\columnwidth} }
		\toprule
		\textbf{Actor}    & A02 \\
		\toprule
		\textbf{Nombre:} 			  & \textbf{Administrador} \\
		\textbf{Versión}              & 1.0    \\
		\textbf{Autor}                & \nombre \\
		\textbf{Descripción}          & Persona que rol de Administrador que interactúa con la aplicación ReservApp para las diferentes opciones de administración que en ella se ofrecen. \\
		\textbf{Tipo}                 & Usuario \\
		\textbf{Objetivo}             & Crear, modificar o anular reservas, crear, modificar o anular convocatorias. \\
		\textbf{Responsabilidades}    & 
		\begin{itemize}
			\tightlist
			\item Gestionar usuarios y establecimientos.
			\item Supervisar reservas del sistema.
			\item Configurar parámetros del sistema.
		\end{itemize}\\
		\textbf{Relaciones con casos de uso} & CU01(\ref{cu:autenticacion-usuario}), CU03(\ref{cu:crear-reserva}), CU04(\ref{cu:modificar-reserva}), CU05(\ref{cu:anular-reserva}), CU06(\ref{cu:gestionar-usuarios}), CU07(\ref{cu:gestionar-establecimientos}), CU08(\ref{cu:ver-reservas-admin}), CU09(\ref{cu:logout}), CU14(\ref{cu:gestionar-perfiles}). \\
		\bottomrule
	\end{tabularx}
	\caption{A01 - Usuario}
	\label{actor:administrador}
\end{table}


\begin{table}[H]
	\centering
	\begin{tabularx}{\linewidth}{ p{0.21\columnwidth} p{0.71\columnwidth} }
		\toprule
		\textbf{Actor}    & A03 \\
		\toprule
		\textbf{Nombre:} 			  & \textbf{Sistema de Email} \\
		\textbf{Versión}              & 1.0    \\
		\textbf{Autor}                & \nombre \\
		\textbf{Descripción}          & Servicio externo para notificaciones por correo. \\
		\textbf{Tipo}                 & Sistema \\
		\textbf{Objetivo}             & Enviar los correos relativos a las reservas y/o convocatorias. \\
		\textbf{Responsabilidades}    & 
		\begin{itemize}
			\tightlist
			\item Enviar notificaciones automáticas.
		\end{itemize}\\
		\textbf{Relaciones con casos de uso} & CU13(\ref{cu:enviar-notificaciones}). \\
		\bottomrule
	\end{tabularx}
	\caption{A01 - Sistema de Email}
	\label{actor:sistema-email}
\end{table}











Una muestra de cómo podría ser una tabla de casos de uso:

% Caso de Uso 1 -> Consultar Experimentos.
\begin{table}[p]
	\centering
	\begin{tabularx}{\linewidth}{ p{0.21\columnwidth} p{0.71\columnwidth} }
		\toprule
		\textbf{CU-1}    & \textbf{Ejemplo de caso de uso}\\
		\toprule
		\textbf{Versión}              & 1.0    \\
		\textbf{Autor}                & Alumno \\
		\textbf{Requisitos asociados} & RF-xx, RF-xx \\
		\textbf{Descripción}          & La descripción del CU \\
		\textbf{Precondición}         & Precondiciones (podría haber más de una) \\
		\textbf{Acciones}             &
		\begin{enumerate}
			\def\labelenumi{\arabic{enumi}.}
			\tightlist
			\item Pasos del CU
			\item Pasos del CU (añadir tantos como sean necesarios)
		\end{enumerate}\\
		\textbf{Postcondición}        & Postcondiciones (podría haber más de una) \\
		\textbf{Excepciones}          & Excepciones \\
		\textbf{Importancia}          & Alta o Media o Baja... \\
		\bottomrule
	\end{tabularx}
	\caption{CU-1 Nombre del caso de uso.}
\end{table}

