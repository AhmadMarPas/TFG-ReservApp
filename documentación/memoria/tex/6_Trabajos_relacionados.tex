\capitulo{6}{Trabajos relacionados}

El diseño de sistemas para gestionar reservas es un campo que ha evolucionado mucho en los últimos años, especialmente gracias al crecimiento de internet y a que las empresas han digitalizado sus gestiones. En este apartado se presenta un análisis de las soluciones más relevantes en el campo de la gestión de reservas de recursos, sistemas de planificación colaborativa y aplicaciones web empresariales que han servido como referencia y contexto para el desarrollo del presente proyecto.

\section{Soluciones comerciales de gestión de reservas}\label{soluciones-comerciales-gestion-reservas}
\subsection{Microsoft Bookings y Microsoft 365}\label{microsoft-booking-microsoft-365}
Microsoft Bookings~\cite{microsoft-booking} representa uno de los sistemas de reservas más extendidos en el ámbito empresarial, integrado nativamente en el ecosistema Microsoft 365~\cite{microsoft-365}. Esta aplicación ha destacado por su capacidad de sincronización automática con Outlook y su integración con Microsoft Teams para reuniones virtuales. Sin embargo, presenta limitaciones significativas en términos de personalización y flexibilidad, especialmente en organizaciones con necesidades específicas en gestión de espacios físicos.
La arquitectura de Microsoft Bookings se rige por un modelo SaaS (Software as a Service)~\cite{saas} que, aunque por un lado reduce la complejidad de mantenimiento, por otro lado limita las posibilidades de personalización y control sobre los datos. En contraposición a esta solución, el enfoque adoptado en el presente TFG permite una mayor flexibilidad de configuración y la posibilidad de adaptarse a requisitos específicos de la organización.

\subsection{Google Workspace y Calendar}\label{google-workspace-calendar}
El sistema de reservas de Google Workspace~\cite{google-workspace}, centrado en Google Calendar~\cite{google-calendar}, destaca por su simplicidad de uso y la excelente integración que establece con otros servicios de Google, pero presenta limitaciones en la gestión granular de recursos y la falta de funcionalidades específicas para la gestión de espacios físicos.
Una característica destacable de Google Calendar es su API robusta que permite integraciones personalizadas. Este aspecto ha influido en la decisión de diseñar la aplicación del presente TFG con una arquitectura modular que facilite futuras integraciones con sistemas externos.

\subsection{Robin y sistemas especializados}\label{robin-sistemas-especializados}
Entre las soluciones actualmente consideradas de nueva generación de sistemas especializados en la gestión de espacios de trabajo se encuentra Robin~\cite{robin}, que se hizo especialmente popular tras la pandemia de COVID-19. Este tipo de sistemas incorporan funcionalidades avanzadas como análisis de ocupación, integración con sistemas IoT y gestión de protocolos sanitarios.
Aunque Robin ofrece funcionalidades avanzadas, el modelo de licenciamiento que conlleva por usuario y su enfoque hacia grandes organizaciones lo hace menos accesible para pequeñas y medianas empresas. Esta limitación no se da en la aplicación desarrollada, ofreciendo una solución propia que se puede adaptar a diferentes tamaños de organización sin restricciones de licenciamiento.

\section{Enfoques académicos y de investigación}\label{enfoques-academicos-investigacion}
\subsection{Optimización de recursos compartidos}\label{optimizacion-recursos-compartidos}
En el ámbito académico, se han desarrollado investigaciones sobre algoritmos de optimización que buscan asignar de forma eficiente espacios compartidos, aprovechando patrones de uso histórico y predicciones de demanda. Entre estos enfoques destaca la idea de la ``reserva inteligente'', capaz de sugerir horarios y espacios de manera automática según la disponibilidad de los participantes y el historial de uso.
Aunque se trata de soluciones técnicamente avanzadas, su puesta en práctica requiere disponer de grandes volúmenes de datos y capacidades de \emph{machine learning} que superan el alcance de la actual propuesta. Aun así, los principios que persiguen optimizar la experiencia del usuario y reducir conflictos han servido como base para diseñar las funciones de validación incluidas en el actual proyecto.

\subsection{Arquitecturas de sistemas colaborativos}\label{arquitecturas-sistemas-colaborativos}
La investigación sobre arquitecturas escalables para sistemas de colaboración empresarial ofrece una base sólida para diseñar aplicaciones web capaces de atender a muchos usuarios al mismo tiempo. Enfoques como la arquitectura de microservicios o el uso de patrones como CQRS (Command Query Responsibility Segregation) aportan ideas útiles para gestionar la consistencia de los datos en sistemas de reservas.
Si bien las arquitecturas distribuidas pueden ser más complejas de lo que un sistema básico requiere, sus principios —como la separación de responsabilidades y la gestión eficiente de la concurrencia— han inspirado la elección de una arquitectura por capas, con una clara división entre las operaciones de lectura y escritura.

\subsection{Estudios de usabilidad en sistemas empresariales}\label{estudios-usuabilidad-sistemas-empresariales}
En los sistemas de gestión empresarial, la experiencia del usuario es importante para que las personas adopten la herramienta y se sientan cómodas usándola. Factores como que el proceso de reserva sea sencillo, que la disponibilidad se muestre de forma clara y que sea fácil hacer cambios sin perderse, pueden marcar la diferencia.
Teniendo esto en cuenta, en este proyecto se han incorporado mejoras como calendarios visuales para ver la disponibilidad de un vistazo, formularios que dan respuesta inmediata a las acciones del usuario y flujos de trabajo más simples para las tareas que se realizan con mayor frecuencia.

\newpage

\section{Posicionamiento del proyecto desarrollado}\label{posicionamiento-proyecto-desarrollado}
Frente a las soluciones analizadas, el presente proyecto se posiciona como una alternativa que combina las siguientes características que se muestran la siguiente tabla comparativa:

\begin{table}[h]
	\centering
	\renewcommand{\arraystretch}{1.5} % Controla el espaciado entre las líneas
	\rowcolors{2}{gray!20}{white} % Alterna colores de fondo en las filas
	\resizebox{\textwidth}{!}{ % Ajusta el tamaño de la tabla al ancho de la página
		\begin{tabular}{m{4.5cm} >{\centering\arraybackslash}m{1.8cm} >{\centering\arraybackslash}m{2.3cm} >{\centering\arraybackslash}m{1.0cm} >{\centering\arraybackslash}m{1.8cm}} % Ajustamos los anchos de las columnas
			\toprule  
			\textbf{Característica} & \textbf{Microsoft Booking} & \textbf{Google Worskpace} & \textbf{Robin} & \textbf{\textit{ReservApp}} \\
			\midrule
			Control y flexibilidad      & Sí & Sí & Sí & Sí \\
			Arquitectura moderna        & Sí & Sí & No & Sí \\
			Enfoque híbrido             & No & No & No & Sí \\
			Escalabilidad progresiva    & Sí & Sí & Sí & Sí \\
			Gestión colaborativa        & Sí & Sí & Sí & Sí \\
			Arquitectura educativa      & No & No & No & Sí \\
			Metodología adaptada        & No & No & No & Sí \\
			Licenciamiento gratuito     & No & Sí & No & Sí \\
			\bottomrule
		\end{tabular}
	}
	\caption{Comparación de aplicaciones}
	\label{herramientasportipodeuso}
\end{table}

\subsection{Limitaciones y contexto}\label{limitaciones-contexto}
Es importante reconocer las limitaciones del proyecto actual:

\begin{itemize}
\tightlist
\item
\textbf{Alcance funcional}: Enfocado a necesidades básicas de gestión de reservas.
\item
\textbf{Funcionalidades avanzadas}: Ausencia de características como IA predictiva o análisis avanzado.
\item
\textbf{Integración}: Limitaciones en conectividad con sistemas empresariales complejos.
\end{itemize}
