\capitulo{2}{Objetivos del proyecto}

En las siguientes secciones se detallan los diferentes objetivos que han promovido o motivado la realización del actual proyecto.

\section{Objetivos generales}\label{objetivos-generales}
Este proyecto busca resolver un problema común en las oficinas: la gestión de las reservas de salas. Por ello, se ha buscado la creación de una solución que no solo mejore la eficiencia en el uso de los espacios, sino que también facilite la vida de los empleados. Con una aplicación web intuitiva, cualquier persona podrá reservar y administrar salas de manera ágil y sin complicaciones.

Con este proyecto no solo se busca solucionar un problema específico dentro de la empresa, sino que se busca la aplicación de un caso práctico en el que entran en juego los conocimientos adquiridos a lo largo de la formación académica para la implementación de una aplicación web para la gestión de recursos compartidos. A lo largo de este desarrollo, se explorarán diferentes herramientas y metodologías para garantizar que la solución final sea escalable, segura y fácil de usar.

\section{Objetivos técnicos}\label{objetivos-tecnicos}
La propuesta de este TFG es la del desarrollo de una aplicación web de tipo cliente-servidor~\cite{client-server-model} que se implementará utilizando, como principales \emph{frameworks}~\cite{framework}, \textit{Spring Boot}~\cite{spring-boot} y \textit{Thymeleaf}~\cite{thymeleaf}, permitiendo la creación de una interfaz dinámica y robusta.

La seguridad fue una prioridad, integrando \textit{Spring Security}~\cite{spring-security} de modo que se garantiza que solo los usuarios autorizados tengan acceso a funcionalidades específicas, protegiendo de esta forma la integridad del sistema.

Para gestionar la evolución del código y su versionado, se empleó \textit{Git} como sistema de control de versiones distribuido, en combinación con un repositorio en \textit{GitHub}. Esto permitió un seguimiento detallado de los cambios realizados.

La estructura de datos fue cuidadosamente diseñada, optando por un modelo relacional normalizado que asegura la coherencia y eficiencia en el manejo de la información. Además, la aplicación siguió la arquitectura MVC (Modelo-Vista-Controlador)~\cite{modelo-vista-controrlador}, favoreciendo de esta forma la modularidad y el mantenimiento del código.

Para el proceso de construcción del software, la gestión de dependencias y la compilación, se utilizó Maven como herramienta de automatización. Además, se implementaron herramientas de integración continua, dispuestas por \textit{GitHub Actions}~\cite{github-actions} e integradas directamente con el repositorio para garantizar un control de calidad constante y automatizado.

Finalmente, en el proyecto se adoptó el marco de trabajo con metodología ágil Scrum, utilizando \textit{Zube}~\cite{zube} como herramienta de gestión, lo que permitió una planificación flexible y adaptable a los cambios, entregando valor de manera iterativa. Además, se implementaron \emph{tests} unitarios y de integración robustos para garantizar la calidad y el correcto funcionamiento de cada componente y del sistema en su conjunto.

\section{Objetivos personales}\label{objetivos-personales}
El proyecto es una apuesta personal por definir una aplicación que pueda tener salida en el entorno laboral, buscando dar servicio a negocios de tipo servicio granular y personalizado a usuarios, y que pueda suponer una salida laboral para mí en el futuro como apuesta de emprendimiento.

Además, con la finalización del TFG, termino también el Grado y es que, tras no haber completado la Ingeniería Técnica en Informática de Gestión en su momento por no haber realizado el Proyecto Fin de Carrera, la realización de este TFG supone la culminación de mi formación académica como ingeniero.