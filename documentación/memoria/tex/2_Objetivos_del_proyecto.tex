\capitulo{2}{Objetivos del proyecto}

En las siguientes secciones se detallan los diferentes objetivos que han promovido o motivado la realización del actual proyecto.

\section{Objetivos generales}\label{objetivos-generales}
El objetivo de este Trabajo de Fin de Grado es diseñar e implementar una solución tecnológica que mejore la gestión de las reservas de salas en el entorno laboral, facilitando la coordinación entre los empleados y mejorando la eficiencia en el uso de estos espacios. Para ello, se desarrollará una aplicación web con una interfaz intuitiva y accesible, que permitirá a los usuarios registrar y administrar reservas de manera ágil y sin ambigüedades.\\

Este proyecto no solo busca solucionar un problema específico dentro de mi empresa, sino que también servirá como un caso práctico de aplicación de tecnologías web en la gestión de recursos compartidos. A lo largo del desarrollo, se explorarán diferentes herramientas y metodologías para garantizar que la solución final sea escalable, segura y fácil de utilizar.

\section{Objetivos técnicos}\label{objetivos-tecnicos}
La propuesta implica el desarrollo de una aplicación web de tipo cliente-servidor que se construirá utilizando, como principales frameworks, Spring Boot y Thymeleaf, lo que permitirá crear una interfaz dinámica y robusta.\\

La seguridad será una prioridad máxima, integrando Spring Security para garantizar que solo los usuarios autorizados tengan acceso a funcionalidades específicas, protegiendo así la integridad del sistema.\\

Para gestionar la evolución del código y el versionado del mismo, se empleará Git como sistema de control de versiones distribuido, en conjunto con un repositorio en GitHub. Esto permitirá un seguimiento detallado de los cambios realizados.\\

La estructura de datos será cuidadosamente diseñada, optando por un modelo relacional normalizado para asegurar la coherencia y eficiencia en el manejo de la información. Además, la aplicación seguirá la arquitectura MVC (Modelo-Vista-Controlador), lo que favorecerá la modularidad y el mantenimiento del código.\\

En cuanto a herramientas para el proceso de construcción del software y la gestión de dependencias y la compilación, se utilizará Maven para automatizar todo este proceso. Se hará uso de herramientas de integración continua como GitHub Actions, integrándolas directamente con el repositorio para mantener un control de calidad constante y automatizado.\\

Finalmente, el proyecto adoptará la metodología ágil Scrum para su desarrollo utilizando Zube como herramienta de gestión de proyectos, lo que permitirá una gestión flexible y adaptable a los cambios, entregando valor de manera iterativa. Se realizarán tests unitarios y de integración exhaustivos para asegurar la calidad y el correcto funcionamiento de cada componente y del sistema en su conjunto.\\

\section{Objetivos personales}\label{objetivos-personales}
El proyecto es una apuesta personal por definir una aplicación que pueda tener salida en el entorno empresarial, buscando dar servicio a negocios de tipo servicio granular y personalizado a usuarios, y que pueda suponer una salida laboral para mí en el futuro como apuesta de emprendimiento.\\

Además, con la finalización del TFG, termino también el Grado y es que, tras no haber completado la Ingeniería Técnica en Informática de Gestión en su momento por no haber realizado el Proyecto Fin de Carrera, la realización de este TFG supone la culminación de mi formación académica como ingeniero.\\
