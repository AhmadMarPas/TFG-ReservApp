\apendice{Especificación de diseño}

\section{Introducción}

En esta sección se detalla la especificación de diseño de la aplicación ``ReservApp''. El objetivo de este documento es proporcionar una visión clara y detallada de la arquitectura de software, el diseño de la base de datos y los flujos de procedimiento de las funcionalidades principales de la aplicación.

``ReservApp'' es una aplicación web desarrollada con el framework Spring Boot, diseñada para la gestión de reservas en diversos tipos de establecimientos. Permite a los usuarios registrarse, buscar establecimientos, realizar reservas y gestionar convocatorias asociadas a dichas reservas.

Este documento está estructurado en las siguientes secciones principales:
\begin{itemize}
    \item \textbf{Diseño de datos:} Describe el modelo de datos de la aplicación, incluyendo las entidades, sus atributos y las relaciones entre ellas. Se adjuntan los diagramas Entidad-Relación y Relacional.
    \item \textbf{Diseño arquitectónico:} Explica la arquitectura de software sobre la que se construye la aplicación, detallando las capas, los patrones de diseño empleados y las tecnologías principales.
    \item \textbf{Diseño procedimental:} Ilustra la secuencia de operaciones para los casos de uso más importantes de la aplicación mediante diagramas de secuencia.
\end{itemize}

El sistema está diseñado siguiendo principios de arquitectura en capas, separación de responsabilidades y buenas prácticas de desarrollo con Spring Framework.

El propósito de esta memoria es servir como guía técnica para el equipo de desarrollo y mantenimiento, facilitando la comprensión del sistema y sirviendo como base para futuras evoluciones del mismo.

\section{Diseño de datos}


\section{Diseño arquitectónico}

\section{Diseño procedimental}



